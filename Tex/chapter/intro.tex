\section{引言}

毕业论文(设计)是专业人才培养方案的重要组成部分, 是学程即将结束时, 学生运用已学知识、理论和技能研究和解决问题的一次综合训练. 毕业论文(设计)在培养大学生探求真理、强化社会意识、进行科学研究基本训练、提高综合实践能力与素质等方面, 具有不可替代的作用, 是教育与生产劳动和社会实践相结合的重要体现, 是培养大学生的创新能力、实践能力和创业精神的重要实践环节.

目前, 学校的教务部门提供了本科生毕业论文(设计)的Word模板, 规定了论文写作的一系列格式. 然而, 在数学论文的排版方面, \LaTeX 系统比Word更有优势. \LaTeX 是\TeX 排版引擎的封装, 具有方便而强大的数学公式排版能力, 很容易生成复杂的专业排版元素, 如脚注、交叉引用、参考文献、目录等. 绝大多数时候, \LaTeX 用户只需专注于一些组织文档结构的基础命令, 无需(或很少)操心文档的版面设计. 为了配合数学专业本科生的毕业论文(设计)写作, 本文作者按照曲阜师范大学本科生毕业论文(设计)格式的要求, 制作了对应的\LaTeX 模板.

本文结构如下: 第2节, 我们给出模板的使用说明, 包含一些重要的注意事项. 第3节, 我们列举了一些常用的\LaTeX 使用例子.

